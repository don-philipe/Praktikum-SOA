\chapter{Einleitung und Aufgabe}

Implementierung eines Webservice in PHP mit dem Slim Framework.

Zusatzaufgabe WS-Security gelöst durch Authentifizierung mittels OAuth2 und Verschlüsselung mittels HTTPS.

\chapter{Funktionen Webservice}

\begin{tabularx}{\columnwidth}{|X|p{1.5cm}|X|p{1.5cm}|}
	\hline
	Name & Method & URL & Access \\
	\hline
	\hline
	Alle verfügbare Fahrradstationen & GET & /stations & public \\
	\hline
	Spezielle Station & GET & /stations/stationID & public \\
	\hline
	Alle verfügbaren Fahrräder & GET & /bikes & public \\
	\hline
	Spezielles Fahrrad & GET & /bikes/bikesID & public \\
	\hline
	Alle Fahrradmodelle & GET & /models & public \\
	\hline
	Spezielles Fahrradmodell & GET & /models/modelID & public \\
	\hline
	Alle Buchungen & GET & /bookings & protected \\
	\hline
	Buchung erstellen & POST & /bookings & protected \\
	\hline
	Einzelne Buchung & GET & /bookings/bookingID & protected \\
	\hline
	Einzelne Buchung stornieren & DELETE & /bookings/bookingID & protected \\
	\hline
	Einzelne Buchung bearbeiten & PUT & /bookings/bookingID & protected \\
	\hline
	Accountinformationen & GET & /account & protected \\
	\hline
\end{tabularx}

\section{GET Methoden}
\subsection{stations}
Liefert eine Liste aller verfügbaren Stationen zurück an denen Fahrräder
zum Verleih zur Verfügung stehen.

\begin{longtable}[c]{@{}lll@{}}
\toprule\addlinespace
Method & URL & Access
\\\addlinespace
\midrule\endhead
GET & /stations & public
\\\addlinespace
\bottomrule
\end{longtable}

\subsubsection{Parameter}\label{parameter}

\begin{longtable}[c]{@{}lll@{}}
\toprule\addlinespace
Name & Required & Description
\\\addlinespace
\midrule\endhead
location & nein & Standort an dem nach Stationen gesucht werden soll,
z.B. ``Dresden'' oder ``Berlin''
\\\addlinespace
model & nein & ID eines bestimmten Fahrradmodells nach dem gefiltert
werden soll, z.B. 105 oder 185
\\\addlinespace
\bottomrule
\end{longtable}

\subsubsection{Request Example}\label{request-example}

\begin{verbatim}
GET /stations?location=Dresden&model=105
\end{verbatim}

\subsubsection{Response}\label{response}

Liefert eine Liste von verfügbaren Stationen zurück, jede Station
besteht aus den folgenden Parametern: * stationId - Die einzigartige
Kennung der Station * name - Der Name der Station * longitude -
Längengrad * latitude - Breitengrad * bikes - Die Anzahl an zum Verleih
zur Verfügung stehenden Fahrräder in der Station * description - Die
ausführliche Beschreibung der Station * pictureUrl - Die URL eines Fotos
von der Station

\subsubsection{Response Examples}\label{response-examples}

\begin{verbatim}
{
   "stations" : [
      {
         "stationId" : 64,
         "name" : "Hauptbahnhof",
         "longitude" : -33.8670522,
         "latitude" : 151.1957362,
         "bikes" : 78,
         "description" : "Tolle Station am Hauptbahnhof, direkt vor dem Ausgang.",
         "pictureUrl" : "www.test.de/station64.jpg"
      },
      {
         "stationId" : 82,
         "name" : "Zoo",
         "longitude" : -33.8670522,
         "latitude" : 151.1957362,
         "bikes" : 108,
         "description" : "Eine Station direkt am Zoo.",
         "pictureUrl" : "www.test.de/station82.jpg"
      }
   ]
}
\end{verbatim}

\subsection{bikes}
Liefert eine Liste aller verfügbaren Fahrräder die zum Verleih zur
Verfügung stehen.

\begin{longtable}[c]{@{}lll@{}}
\toprule\addlinespace
Method & URL & Access
\\\addlinespace
\midrule\endhead
GET & /bikes & public
\\\addlinespace
\bottomrule
\end{longtable}

\subsubsection{Parameter}\label{parameter}

\begin{longtable}[c]{@{}lll@{}}
\toprule\addlinespace
Name & Required & Description
\\\addlinespace
\midrule\endhead
location & ja & Standort an dem nach verfügbaren Fahrrädern gesucht werden soll, z.B. ``Dresden''
\\\addlinespace
radius & nein & Radius in dem um die location gesucht werden soll in Meter (default: 2500), z.B. 5000
\\\addlinespace
modelId & nein & Die ID eines bestimmten Fahrradmodells nach dem gefiltert werden soll, z.B. 105
\\\addlinespace
stationId & nein & Die ID einer bestimmten Verleihstation nach der gefiltert werden soll, z.B. 46
\\\addlinespace
\bottomrule
\end{longtable}

\subsubsection{Request Example}\label{request-example}

\begin{verbatim}
GET /bikes?location=Dresden&radius=5000&modelId=105&stationId=4
\end{verbatim}

\subsubsection{Response}\label{response}

Liefert eine Liste von verfügbaren Fahrrädern, jedes Fahrrad besteht aus
den folgenden Parametern: * bikeId - Die einzigartige Kennung des
Fahrrads * modelId - Die einzigartige Kennung des Fahrradmodells * price
- Der Preis für 15 Minuten in Cent. * longitude - Längengrad * latitude
- Breitengrad * stationId - Falls das Fahrrad in einer Verleihstation
steht, ist dies die eindeutige Kennung der Station * distance - Die
Entfernung zum gewählten Standort

\subsubsection{Response Examples}\label{response-examples}

\begin{verbatim}
{
   "bikes" : [
      {
         "bikeId" : 46,
         "modelId" : 105,
         "price" : 100,
         "longitude" : -33.8670522,
         "latitude" : 151.1957362,
         "stationId" : 4,
         "distance": 2800
      },
      {
         "bikeId" : 34,
         "modelId" : 105,
         "price" : 150,
         "longitude" : -33.8670522,
         "latitude" : 151.1957362,
         "stationId" : 4 ,
         "distance": 2800
      }
   ]
}
\end{verbatim}

\subsection{models}
Liefert eine Liste aller verfügbaren Fahrradmodelle.

\begin{longtable}[c]{@{}lll@{}}
\toprule\addlinespace
Method & URL & Access
\\\addlinespace
\midrule\endhead
GET & /models & public
\\\addlinespace
\bottomrule
\end{longtable}

\subsubsection{Parameter}\label{parameter}

\begin{longtable}[c]{@{}lll@{}}
\toprule\addlinespace
Name & Required & Description
\\\addlinespace
\midrule\endhead
location & nein & Standort an dem nach verfügbaren Fahrrädern gesucht
werden soll, z.B. ``Dresden'' oder ``Berlin''
\\\addlinespace
stationId & nein & Die eindeutige Kennung einer bestimmten
Verleihstation nach der gefiltert werden soll, z.B. 46 oder 4
\\\addlinespace
\bottomrule
\end{longtable}

\subsubsection{Request Example}\label{request-example}

\begin{verbatim}
GET /models?location=Dresden&stationId=4
\end{verbatim}

\subsubsection{Response}\label{response}

Liefert eine Liste von verfügbaren Fahrradmodellen, jedes Modell besteht
aus den folgenden Parametern: * modelId - Die einzigartige Kennung des
Fahrradmodells * name - Der Name des Fahrradmodells * description - Eine
Beschreibung des Modells * pictureUrl - Die URL eines Fotos, welches das
Fahrradmodell zeigt * bikes - Die Anzahl an zum Verleih zur Verfügung
stehenden Fahrräder dieses Modells

\subsubsection{Response Examples}\label{response-examples}

\begin{verbatim}
{
   "models" : [
      {
         "modelId" : 105,
         "name" : "Rennrad",
         "description" : "Bestens geeignet für das Fahren auf dem Asphalt.",
         "pictureUrl" : "www.test.de/model105",
         "bikes" : 78
      },
      {
         "modelId" : 68,
         "name" : "Kinderfahrrad",
         "description" : "Geeignet für Kinder zwischen 6 und 9 Jahren.",
         "pictureUrl" : "www.test.de/model68",
         "bikes" : 8
      }
   ]
}
\end{verbatim}

\subsection{account}
Liefert die Accountinformationen des Nutzers.

\begin{longtable}[c]{@{}lll@{}}
\toprule\addlinespace
Method & URL & Access
\\\addlinespace
\midrule\endhead
GET & /account & protected
\\\addlinespace
\bottomrule
\end{longtable}

\subsubsection{Request Example}\label{request-example}

\begin{verbatim}
GET /account
\end{verbatim}

\subsubsection{Response}\label{response}

Liefert die Accountinformationen des Nutzers, ein Account besteht aus
den folgenden Parametern: * accountId - Die einzigartige Kennung des
Nutzers * email - Die E-Mail Adresse des Nutzers

\subsubsection{Response Examples}\label{response-examples}

\begin{verbatim}
{
   "accountId" : 1696,
   "email" : "test@test.com"
}
\end{verbatim}

\subsection{bookings}
Liefert eine Liste aller getätigten Buchungen.

\begin{longtable}[c]{@{}lll@{}}
\toprule\addlinespace
Method & URL & Access
\\\addlinespace
\midrule\endhead
GET & /bookings & protected
\\\addlinespace
\bottomrule
\end{longtable}

\subsubsection{Request Example}\label{request-example}

\begin{verbatim}
GET /bookings
\end{verbatim}

\subsubsection{Response}\label{response}

Liefert eine Liste aller getätigten Buchungen, jede Buchung besteht aus
den folgenden Parametern: * bookingId - Die einzigartige Kennung der
Buchung * bikeId - Die einzigartige Kennung des gebuchten Fahrrads *
date - Das Datum der Buchung

\subsubsection{Response Examples}\label{response-examples}

\begin{verbatim}
{
   "bookings" : [
      {
         "bookingId" : 1682,
         "bikeId" : 105,
         "date" : "2013-12-01 18:44:36"
      },
      {
         "bookingId" : 1696,
         "bikeId" : 168,
         "date" : "2013-12-01 18:45:24"
      }
   ]
}
\end{verbatim}

\section{POST Methoden}
\subsection{bookings}
Erstellt eine einzelne Buchung.

\begin{longtable}[c]{@{}lll@{}}
\toprule\addlinespace
Method & URL & Access
\\\addlinespace
\midrule\endhead
POST & /bookings & protected
\\\addlinespace
\bottomrule
\end{longtable}

\subsubsection{Request Example}\label{request-example}

\begin{verbatim}
POST /bookings
Params: {
   "id": 168
}
\end{verbatim}

\subsubsection{Parameter}\label{parameter}

\begin{longtable}[c]{@{}lll@{}}
\toprule\addlinespace
Name & Required & Description
\\\addlinespace
\midrule\endhead
id & ja & Die einzigartige Kennung des Fahrrads, z.B. 168
\\\addlinespace
\bottomrule
\end{longtable}

\subsubsection{Response}\label{response}

Liefert eine einzelne getätigte Buchung, die Buchung besteht aus den
folgenden Parametern: - 
\begin{itemize}
\item id - Die einzigartige Kennung der Buchung
\item bikeId - Die einzigartige Kennung des gebuchten Fahrrads
\item date - Der Zeitpunkt der Buchung
\end{itemize}

\subsubsection{Response Examples}\label{response-examples}

\begin{verbatim}
{
   	"bookingId" : 1696,
   	"bikeId" : 168,
   	"date" : "2013-12-01 18:45:24"
}
\end{verbatim}

\section{DELETE Methoden}
\subsection{bookings}
Storniert eine einzelne Buchung.

\begin{longtable}[c]{@{}lll@{}}
\toprule\addlinespace
Method & URL & Access
\\\addlinespace
\midrule\endhead
DELETE & /bookings/bookingId & protected
\\\addlinespace
\bottomrule
\end{longtable}

\subsubsection{Request Example}\label{request-example}

\begin{verbatim}
DELETE /bookings/1696
\end{verbatim}

\section{PUT Methoden}
\subsection{bookings}
Ändert eine einzelne Buchung.

WICHTIG: Content-Type: application/x-www-form-urlencoded

\begin{longtable}[c]{@{}lll@{}}
\toprule\addlinespace
Method & URL & Access
\\\addlinespace
\midrule\endhead
PUT & /bookings/bookingId & protected
\\\addlinespace
\bottomrule
\end{longtable}

\subsubsection{Request Example}\label{request-example}

\begin{verbatim}
PUT /bookings/bookingId
Params: {
   "bikeId": 168
}
\end{verbatim}

\subsubsection{Parameter}\label{parameter}

\begin{longtable}[c]{@{}lll@{}}
\toprule\addlinespace
Name & Required & Description
\\\addlinespace
\midrule\endhead
bikeId & nein & Die einzigartige Kennung des Fahrrads, z.B. 168
\\\addlinespace
\bottomrule
\end{longtable}

\subsubsection{Response}\label{response}

Liefert die geänderte Buchung, die Buchung besteht aus den folgenden
Parametern: - bookingId - Die einzigartige Kennung der Buchung - bikeId
- Die einzigartige Kennung des gebuchten Fahrrads - date - Der Zeitpunkt
an dem die Buchung getätigt wurde

\subsubsection{Response Examples}\label{response-examples}

\begin{verbatim}
{
   "bookingId" : 1696,
   "bikeId" : 168,
   "date" : "2013-12-01 18:45:24"
}
\end{verbatim}


\chapter{Webclient}

Der Webclient ist so konzipiert, dass prinzipiell auch mobile Endgeräte unterstützt werden.
Er bietet alle Funktionen die im vorhergehenden Kapitel beschrieben sind.

\begin{figure}
        \centering
	\includegraphics[height=80mm]{pics/bikesharing_search.png}
\end{figure}

\chapter{Zusatzaufgabe WS-Security}

\section{OAuth2}

Das Architekturkonzept von OAuth2 wird durch folgendes Schema verdeutlicht.

\begin{figure}
        \centering
	\includegraphics[height=80mm]{pics/Architektur.png}
\end{figure}

\section{Implementierung}

\chapter{Fazit}

Die Implementierung des Webclient ist nach Vorlage einer durchdachten API gut machbar.

Das Slim-Framework war eine gute Wahl, da die Verwendung sehr einfach und fehlerfrei verlief.
Die OAuth-Middleware hat leider nicht funktioniert.

Implementierung eines OAuth-Servers ist relativ kompliziert.


\chapter{Quellen}

