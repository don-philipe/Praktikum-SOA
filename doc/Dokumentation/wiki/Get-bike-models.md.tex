Liefert eine Liste aller verfügbaren Fahrradmodelle.

\begin{longtable}[c]{@{}lll@{}}
\toprule\addlinespace
Method & URL & Access
\\\addlinespace
\midrule\endhead
GET & /models & public
\\\addlinespace
\bottomrule
\end{longtable}

\subsubsection{Parameter}\label{parameter}

\begin{longtable}[c]{@{}lll@{}}
\toprule\addlinespace
Name & Required & Description
\\\addlinespace
\midrule\endhead
location & nein & Standort an dem nach verfügbaren Fahrrädern gesucht werden soll, z.B. ``Dresden''
\\\addlinespace
stationId & nein & Die ID einer bestimmten Verleihstation nach der gefiltert werden soll, z.B. 46
\\\addlinespace
\bottomrule
\end{longtable}

\subsubsection{Request Example}\label{request-example}

\begin{verbatim}
GET /models?location=Dresden&stationId=4
\end{verbatim}

\subsubsection{Response}\label{response}

Liefert eine Liste von verfügbaren Fahrradmodellen, jedes Modell besteht
aus den folgenden Parametern: 
\begin{itemize}
\item id - Die einzigartige Kennung des Fahrradmodells
\item name - Der Name des Fahrradmodells
\item description - Die ausführliche Beschreibung des Modells
\item picture - Die URL eines Fotos, welches das Fahrradmodell zeigt
\item  bikes - Die Anzahl an zum Verleih zur Verfügung stehenden Fahrräder dieses Modells
\end{itemize}

\subsubsection{Response Examples}\label{response-examples}

\begin{verbatim}
{
         "id" : 105,
         "name" : "Rennrad",
         "description" : "Bestens geeignet für das Fahren auf dem Asphalt.",
         "picture" : "www.test.de/model105",
         "bikes" : 78
},
{
         "id" : 68,
         "name" : "Kinderfahrrad",
         "description" : "Geeignet für Kinder zwischen 6 und 9 Jahren.",
         "picture" : "www.test.de/model68",
         "bikes" : 8
}
\end{verbatim}
