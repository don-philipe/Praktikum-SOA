Gibt eine einzelne Station mit der Kennung \emph{stationId} zurück.

\begin{longtable}[c]{@{}lll@{}}
\toprule\addlinespace
Method & URL & Access
\\\addlinespace
\midrule\endhead
GET & /stations/stationId & public
\\\addlinespace
\bottomrule
\end{longtable}

\subsubsection{Request Example}\label{request-example}

\begin{verbatim}
GET /stations/4
\end{verbatim}

\subsubsection{Response}\label{response}

Gibt eine einzelne Station zurück, die Station besteht aus den folgenden
Parametern: * stationId - Die einzigartige Kennung der Station * name -
Der Name der Station * longitude - Längengrad * latitude - Breitengrad *
bikes - Die Anzahl an zum Verleih zur Verfügung stehenden Fahrräder in
der Station * description - Die ausführliche Beschreibung der Station *
pictureUrl - Die URL eines Fotos von der Station

\subsubsection{Response Examples}\label{response-examples}

\begin{verbatim}
{
   "stationId" : 4,
   "name" : "Hauptbahnhof",
   "longitude" : -33.8670522,
   "latitude" : 151.1957362,
   "bikes" : 78,
   "description" : "Tolle Station am Hauptbahnhof, direkt vor dem Ausgang.",
   "pictureUrl" : "www.test.de/station4.jpg"
}
\end{verbatim}
