Erstellt eine einzelne Buchung.

\begin{longtable}[c]{@{}lll@{}}
\toprule\addlinespace
Method & URL & Access
\\\addlinespace
\midrule\endhead
POST & /bookings & protected
\\\addlinespace
\bottomrule
\end{longtable}

\subsubsection{Request Example}\label{request-example}

\begin{verbatim}
POST /bookings
Params: {
   "bikeId": 168
}
\end{verbatim}

\subsubsection{Parameter}\label{parameter}

\begin{longtable}[c]{@{}lll@{}}
\toprule\addlinespace
Name & Required & Description
\\\addlinespace
\midrule\endhead
bikeId & ja & Die einzigartige Kennung des Fahrrads, z.B. 168
\\\addlinespace
\bottomrule
\end{longtable}

\subsubsection{Response}\label{response}

Liefert eine einzelne getätigte Buchung, die Buchung besteht aus den
folgenden Parametern: - bookingId - Die einzigartige Kennung der Buchung
- bikeId - Die einzigartige Kennung des gebuchten Fahrrads - date - Der
Zeitpunkt an dem die Buchung getätigt wurde

\subsubsection{Response Examples}\label{response-examples}

\begin{verbatim}
{
   "bookingId" : 1696,
   "bikeId" : 168,
   "date" : "2013-12-01 18:45:24"
}
\end{verbatim}
