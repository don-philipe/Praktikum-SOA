\begin{frame}
	\frametitle*{Inhalt}
	\tableofcontents
\end{frame}

\section{Teamvorstellung}
\begin{frame}
	\frametitle*{Teamvorstellung}
	\begin{itemize}
		\item Tobias Hafermalz
		\item Philipp Thöricht
	\end{itemize}
\end{frame}

\section{Szenario}
\begin{frame}
	\frametitle*{Szenario: Bike-Sharing} 
	Die traditionelle Fahrradvermietung entspricht nicht mehr den Bedürfnissen der Menschen in mittelgroßen und großen Städten.
	Man möchte unkompliziert an ein Fahrrad kommen, ohne weit dafür fahren/gehen zu müssen.
	Wie beim Car-Sharing soll es zum einen Stationen geben an denen eine bestimmte Menge an Fahrrädern bereitgestellt wird, andererseits braucht man ein Fahrrad nicht zwangsläufig zu dieser Station zurückzubringen, sondern kann es, innerhalb gewisser Zonen, abstellen, so dass andere es wieder nutzen können.
\end{frame}

\begin{frame}
	\frametitle*{Szenario - Features}
	\begin{itemize}
		\item Standortauswahl
		\item Fahrradauswahl 
		\item Anzahl
		\item Preise
		\item Suche
		\item Dauer
		\item Buchung
		\item Profil
	\end{itemize}
\end{frame}

\section{Technologien}
\begin{frame}
	\frametitle*{Technologien}
	\begin{itemize}
		\item PHP
		\item Git - auf GitHub gehostet
		\item Framework ???
		\item WebClient
	\end{itemize}
\end{frame}

\section{Vertiefung}
\begin{frame}
	\frametitle*{Vertiefung}
	\begin{itemize}
		\item Sicherheit
			\begin{itemize}
				\item API Key
				\item HTTPS
				\item HTTP Basic Auth
			\end{itemize}
	\end{itemize}
\end{frame}

\section{Quellen}
\begin{frame}
	\frametitle*{Quellen}
	\begin{itemize}
		\item \url{https://www.tu-braunschweig.de/winfo/researchprojects/bikesharing/index.html}
	\end{itemize}
	%\bibliography{../literatur}
\end{frame}
